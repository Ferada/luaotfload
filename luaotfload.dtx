% \iffalse meta-comment
%
% Copyright (C) 2009 by Elie Roux <elie.roux@telecom-bretagne.eu>
%
% This work is under the CC0 license.
%
% This work consists of the main source file luaotfload.dtx
% and the derived files
%    luaotfload.sty, luaotfload.lua
%
% Unpacking:
%    tex luaotfload.dtx
%
% Documentation:
%    pdflatex luaotfload.dtx
%
%    The class ltxdoc loads the configuration file ltxdoc.cfg
%    if available. Here you can specify further options, e.g.
%    use A4 as paper format:
%       \PassOptionsToClass{a4paper}{article}
%
%
%
%<*ignore>
\begingroup
  \def\x{LaTeX2e}%
\expandafter\endgroup
\ifcase 0\ifx\install y1\fi\expandafter
         \ifx\csname processbatchFile\endcsname\relax\else1\fi
         \ifx\fmtname\x\else 1\fi\relax
\else\csname fi\endcsname
%</ignore>
%<*install>
\input docstrip.tex
\Msg{************************************************************************}
\Msg{* Installation}
\Msg{* Package: luaotfload 2010/01/08 v1.07 ConTeXt font loading system}
\Msg{************************************************************************}

\keepsilent
\askforoverwritefalse

\let\MetaPrefix\relax

\preamble
This is a generated file.

Copyright (C) 2009 by Elie Roux <elie.roux@telecom-bretagne.eu>

This work is under the CC0 license.

This work consists of the main source file luaotfload.dtx
and the derived files
    luaotfload.sty, luaotfload.lua

\endpreamble

\let\MetaPrefix\DoubleperCent


\generate{%
  \usedir{tex/luatex/luaodfload}%
  \file{luaotfload.sty}{\from{luaotfload.dtx}{package}}%
}

% The following hacks are to generate a lua file with lua comments starting by
% -- instead of %%

\def\MetaPrefix{-- }

\def\luapostamble{%
  \MetaPrefix^^J%
  \MetaPrefix\space End of File `\outFileName'.%
}

\def\currentpostamble{\luapostamble}%

\generate{%
  \usedir{tex/luatex/luaotfload}%
  \file{luaotfload.lua}{\from{luaotfload.dtx}{lua}}%%
}

\obeyspaces
\Msg{************************************************************************}
\Msg{*}
\Msg{* To finish the installation you have to move the following}
\Msg{* files into a directory searched by TeX:}
\Msg{*}
\Msg{*     luaotfload.sty, luaotfload.lua}
\Msg{*}
\Msg{* Happy TeXing!}
\Msg{*}
\Msg{************************************************************************}

\endbatchfile
%</install>
%<*ignore>
\fi
%</ignore>
%<*driver>
\NeedsTeXFormat{LaTeX2e}
\ProvidesFile{luaotfload.drv}%
  [2010/01/08 v1.07 ConTeXt font loading system]%
\documentclass{ltxdoc}
\usepackage{metalogo}
\usepackage[colorlinks=true]{hyperref}
\EnableCrossrefs
\CodelineIndex
\begin{document}
  \DocInput{luaotfload.dtx}%
\end{document}
%</driver>
% \fi
%
% \CheckSum{0}
%
% \CharacterTable
%  {Upper-case    \A\B\C\D\E\F\G\H\I\J\K\L\M\N\O\P\Q\R\S\T\U\V\W\X\Y\Z
%   Lower-case    \a\b\c\d\e\f\g\h\i\j\k\l\m\n\o\p\q\r\s\t\u\v\w\x\y\z
%   Digits        \0\1\2\3\4\5\6\7\8\9
%   Exclamation   \!     Double quote  \"     Hash (number) \#
%   Dollar        \$     Percent       \%     Ampersand     \&
%   Acute accent  \'     Left paren    \(     Right paren   \)
%   Asterisk      \*     Plus          \+     Comma         \,
%   Minus         \-     Point         \.     Solidus       \/
%   Colon         \:     Semicolon     \;     Less than     \<
%   Equals        \=     Greater than  \>     Question mark \?
%   Commercial at \@     Left bracket  \[     Backslash     \\
%   Right bracket \]     Circumflex    \^     Underscore    \_
%   Grave accent  \`     Left brace    \{     Vertical bar  \|
%   Right brace   \}     Tilde         \~}
%
% \GetFileInfo{luaotfload.drv}
%
% \title{The \textsf{luaotfload} package}
% \date{2010/01/08 v1.07}
% \author{Elie Roux \\ \texttt{elie.roux@telecom-bretagne.eu}}
%
% \maketitle
%
% \begin{abstract}
% Con\TeX t font loading system, providing the possibility to load OTF fonts
% with a lot of features, and the \XeTeX\ font loading syntax.
% \end{abstract}
%
% \tableofcontents
%
% \section{Documentation}
% 
% \subsection{Introduction}
%
% Font management and installation has always been painful with \TeX\ (and
% even more with \LaTeX ). A lot of files are needed for one font (tfm, pfb,
% map, fd, vf), and they are limited to 256 characters. But the font world has
% evolved since, and new standard types of fonts have appeared, like
% \textsf{truetype} or \textsf{opentype} fonts. These fonts can contain a lot
% of characters, and have some functionalities (ligatures, old-style numbers,
% small capitals, etc.). They are everywhere, as the system fonts and most
% modern text softwares fonts are of this type. Until now the (almost) only
% way to use them with \TeX\ was to use them with \XeTeX .
%
% Unlike \XeTeX , Lua\TeX\ does not provide facilities for these fonts by
% default, but it provides a way to hook lua code in some points of the \TeX\
% algorithm, for instance we can improve the font loading system; this is what
% we do in this package.
%
% This package is quite low-level, and should be loaded directly in the macro
% package, like it is in Con\TeX t. Sadly, Plain and \LaTeX\ are frozen and
% it's even impossible to adapt them to the new engines. 
%
% \subsection{Loading fonts}
%
% \textsf{luaotfload} supports \XeTeX -like font syntax which looks like:
%
% \begin{center}
% |\font\foo="|\meta{prefix}|:|\meta{font name}|:|\meta{font features}|"| \meta{\TeX\ font features}
% \end{center}
%
% \subsubsection{Prefix}
%
% Prefixes are either |file| or |name|; |file| prefix forces searching font
% based on filename, while a |name| prefix will result in consulting name
% database. Name database can be generated with\footnote{|mtxrun| tool and
% scripts are part of Con\TeX t}:
%
% \begin{center}
% |mtxrun --script fonts --names|
% \footnote{|mtxrun --script fonts --reload --simple| in recent
% Con\TeX t releases}
% \end{center}
%
% \noindent Which will generate a file named |luatex-fonts-names.lua| that is
% to be placed somewhere in your \textsc{texmf} tree where \textsf{kpathsea} can
% find it.
%
% \subsubsection{Font name}
%
% Font name can be either a font filename or actual font name. If no prefix is
% specified; it is assumed to be a filename. \XeTeX-like syntax of using square
% brackets to select fonts from \textsc{texmf} tree is olso supported;
% |\font\foo=[foo.ttf]:+liga|.
%
% \subsubsection{Font features}
%
% By default \textsf{luaotfload} doesn't enable any font features, font features
% must be explicitly enabled, |\font\foo=foo.ttf:+liga| will load |foo.ttf| font
% and enable \texttt{liga} feature. In the same way font features can be
% deactivated by prefixing it with minus sign; |\font\foo=foo.ttf:+liga;-liga|
% will load \texttt{foo.ttf} without any features enabled.
%
% \paragraph{Modes}
% \textsf{luaotfload} has two OpenType processing modes; \texttt{base} mode
% which enables only a subset of OpenType features and works by mapping those
% features to traditional TeX ligaturing and kerning mechanisms and is a bit
% faster, and \texttt{node} mode which, hopefully, supports OpenType fully and
% works by direct processing of node list at lua end and is a bit slower.
% Note that, nside math mode only \texttt{base} is supported.
%
% Modes can be selected using \texttt{mode} keyword and it must come before any
% other font feature, e.g. |\font\foo=foo.ttf:mode=node;+liga|.
%
% \paragraph{Additional font features}
% \textsf{luaotfload} defines some additional font feature not defined in
% OpenType, currently three features are defined:
%
% \begin{itemize}
% \item[-] \texttt{anum}: Arabic numbers
% \item[-] \texttt{tlig}: \TeX\ ligatures
% \item[-] \texttt{trep}: \TeX\ replacements
% \end{itemize}
%
% \subsubsection{Examples}
% \begin{verbatim}
% \font\testa=file:lmroman10-regular                  at 12pt \testa \input tufte \par
% \font\testb=file:lmroman12-regular:+liga;           at 24pt \testb effe flink fietsen \par
% \font\testc=file:lmroman12-regular:mode=node;+liga; at 24pt \testc effe flink fietsen \par
% \font\testd=name:lmroman10bold                      at 12pt \testd a bit bold \par
%
% \font\teste=[lmroman12-regular]:+liga at 30pt \teste text
% \font\testf=[lmroman12-regular]       at 40pt \testf more text
% \end{verbatim}
%
% \subsection{Con\TeX t files needed}
%
% This package is a wrapper for several files taken from the Con\TeX t macro
% package. The philosophy is to let Con\TeX t do all the implementation and
% update these files from time to time. To do so we did not modify the files
% taken from Con\TeX t, we only changed their names to prevent name clashes.
% You can thus update the font system of this package simply by updating the
% files taken from Con\TeX t, without (theorically) changing the \texttt{.sty}
% file nor the main \texttt{.lua} file.
%
% The Con\TeX t files are renamed by adding the prefix \texttt{otfl-} to them
% (\texttt{otfl} as \texttt{OTF L}oad).The files are:
%
% \begin{itemize}
% \item \texttt{luat-dum.lua}
% \item \texttt{data-con.lua}
% \item \texttt{node-ini.lua}
% \item \texttt{node-inj.lua}
% \item \texttt{node-fnt.lua}
% \item \texttt{node-dum.lua}
% \item \texttt{font-ini.lua}
% \item \texttt{font-tfm.lua}
% \item \texttt{font-cid.lua}
% \item \texttt{font-ott.lua}
% \item \texttt{font-otf.lua}
% \item \texttt{font-otd.lua}
% \item \texttt{font-oti.lua}
% \item \texttt{font-otb.lua}
% \item \texttt{font-otn.lua}
% \item \texttt{font-ota.lua}
% \item \texttt{font-otc.lua}
% \item \texttt{font-def.lua}
% \item \texttt{font-xtx.lua}
% \item \texttt{font-map.lua}
% \item \texttt{font-dum.lua}
% \end{itemize}
% 
% \subsection{Troubleshooting}
%
% If you encounter problems with some fonts, please first update to the latest version of this package before reporting a bug, as this package is under active development.
%
% A very common problem is the lack of features for some otf fonts even when specified. It can be related to the fact that some fonts do not provide features for the |dflt| script, which is the default one in this package, so you may have to specify the script in the command line, for example:
%
% |\font\myfont = MyFont.otf:script=latn;+liga;|
%
% \section{\texttt{luaotfload.lua}}
%
% \iffalse
%<*lua>
% \fi
%
%    First some usual initializations.
%
%    \begin{macrocode}
luaotfload          = { }

luaotfload.module = {
    name          = "luaotfload",
    version       = 1.06,
    date          = "2009/12/29",
    description   = "ConTeXt font loading system.",
    author        = "Elie Roux & Hans Hagen",
    copyright     = "Elie Roux",
    license       = "CC0"
}

luatextra.provides_module(luaotfload.module)

%    \end{macrocode}
%
%    We load the Con\TeX t files with this function. It automatically adds the
%    \texttt{otfl-} prefix to it, so that we call it with the actual Con\TeX t
%    name.
%
%    Some Con\TeX t files introduce incompatibilities with old Lua\TeX s, thus
%    the old versions of these files are kept under the name |otfl-compat-...|
%    and are called according to the version. The only file in this case
%    is a |font-otf.lua| that, in version 2009.11.26 of Con\TeX t generates
%    invalid PDFs with Lua\TeX{} older than version 0.45.
%
%    \begin{macrocode}


function luaotfload.loadmodule(name, compat)
    local tofind
    if compat then
        if tex.luatexversion < 45 then
            tofind = 'otfl-compat-'..name
        else
            tofind = 'otfl-'..name
        end
    else
        tofind = 'otfl-'..name
    end
    local found = kpse.find_file(tofind,"tex")
    if not found then
      luatextra.module_error('luaotfload', string.format('file %s not found.', tofind))
      return
    end
    luatextra.module_log('luaotfload', "loading file "..found)
    dofile(found)
end

%    \end{macrocode}
%
%    The following functions are made to map Con\TeX t functions to
%    \textsf{luaextra} functions.
%
%    \begin{macrocode}

string.strip = string.stripspaces

file = fpath
file.extname = fpath.suffix

%    \end{macrocode}
%
%    These are small functions that are not already in \textsf{luatextra}.
%
%    \begin{macrocode}

local splitters_s, splitters_m = { }, { }

function lpeg.splitat(separator,single)
    local splitter = (single and splitters_s[separator]) or splitters_m[separator]
    if not splitter then
        separator = lpeg.P(separator)
        if single then
            local other, any = lpeg.C((1 - separator)^0), lpeg.P(1)
            splitter = other * (separator * lpeg.C(any^0) + "")
            splitters_s[separator] = splitter
        else
            local other = lpeg.C((1 - separator)^0)
            splitter = other * (separator * other)^0
            splitters_m[separator] = splitter
        end
    end
    return splitter
end

function table.compact(t)
    if t then
        for k,v in next, t do
            if not next(v) then
                t[k] = nil
            end
        end
    end
end

function table.sortedhashkeys(tab) -- fast one
    local srt = { }
    for key,_ in next, tab do
        srt[#srt+1] = key
    end
    table.sort(srt)
    return srt
end

function table.reverse_hash(h)
    local r = { }
    for k,v in next, h do
        r[v] = string.lower(string.gsub(k," ",""))
    end 
    return r
end 

function table.reverse(t)
    local tt = { }
    if #t > 0 then
        for i=#t,1,-1 do
            tt[#tt+1] = t[i]
        end
    end
    return tt
end

%    \end{macrocode}
%
%    We start loading some lua files. These two are some code not used by
%    Con\TeX t at all that allow other modules to be used, it provides some
%    low-level Con\TeX t functions.
%
%    \begin{macrocode}

luaotfload.loadmodule('luat-dum.lua') -- not used in context at all
luaotfload.loadmodule('data-con.lua') -- maybe some day we don't need this one

%    \end{macrocode}
%
%    This one is for node support.
%
%    \begin{macrocode}

luaotfload.loadmodule('node-ini.lua')

%    \end{macrocode}
%
%    By default Con\TeX t takes some private attributes for internal use. With
%    Plain and \LaTeX\ we can't do so, we use |\newluaattribute|. This
%    functions overrides a function defined in the previous module that
%    returns the number of a private attribute. We allocate new attributes in
%    the \texttt{.sty} file, and this function returns their number. Like this
%    we don't need any private attribute, and this package is compatible with
%    the others. We use the |otfl@| prefix for attributes.
%
%    \begin{macrocode}

function attributes.private(name)
    local number = tex.attributenumber['otfl@'..name]
    if not number then 
        luatextra.module_error('luaotfload', string.format('asking for attribute %s, but not declared. Please report to the maintainer of luaotfload.', name))
    end
    return number
end

%    \end{macrocode}
%
%    Some more modules. We don't load neither \texttt{font-enc.lua} nor
%    \texttt{font-afm.lua} as it will never be used here.
%    
%    We also remove a warning from \texttt{node-fnt.lua} as it not relevant
%    with Lua\TeX{}tra.
%
%    \begin{macrocode}

tex.attribute[0] = 0

luaotfload.loadmodule('node-res.lua')
luaotfload.loadmodule('node-inj.lua')
luaotfload.loadmodule('node-fnt.lua')
luaotfload.loadmodule('node-dum.lua')

luaotfload.loadmodule('font-ini.lua')
luaotfload.loadmodule('font-tfm.lua')
luaotfload.loadmodule('font-cid.lua')
luaotfload.loadmodule('font-ott.lua')
luaotfload.loadmodule('font-otf.lua')
luaotfload.loadmodule('font-otd.lua')
luaotfload.loadmodule('font-oti.lua')
luaotfload.loadmodule('font-otb.lua')
luaotfload.loadmodule('font-otn.lua')
luaotfload.loadmodule('font-ota.lua')
luaotfload.loadmodule('font-otc.lua')
luaotfload.loadmodule('font-def.lua')
luaotfload.loadmodule('font-xtx.lua')
luaotfload.loadmodule('font-map.lua')
luaotfload.loadmodule('font-dum.lua')
luaotfload.loadmodule('font-clr.lua')

%    \end{macrocode}
%
%    This is a small patch that prevents errors in some \LaTeX\ files.
%
%    \begin{macrocode}

fonts.enc.known = {}

%    \end{macrocode}
%
%    Finally two functions 
%
%    \begin{macrocode}

function luaotfload.register_callbacks()
    callback.add('pre_linebreak_filter', nodes.simple_font_handler, 'luaotfload.pre_linebreak_filter')
    callback.add('hpack_filter',         nodes.simple_font_handler, 'luaotfload.hpack_filter')
    callback.reset('define_font')
    callback.add('define_font', fonts.define.read, 'luaotfload.define_font', 1)
    callback.add('find_vf_file', fonts.vf.find, 'luaotfload.find_vf_file')
end

function luaotfload.unregister_callbacks()
    callback.remove('pre_linebreak_filter', 'luaotfload.pre_linebreak_filter')
    callback.remove('hpack_filter', 'luaotfload.hpack_filter')
    callback.remove('define_font', 'luaotfload.define_font')
    callback.remove('find_vf_file', 'luaotfload.find_vf_file')
end
%    \end{macrocode}
% \iffalse
%</lua>
% \fi
%
% \section{\texttt{luaotfload.sty}}
%
% \iffalse
%<*package>
% \fi
%
%    Classical Plain+\LaTeX\ package initialization.
%
%    \begin{macrocode}
\csname ifluaotfloadloaded\endcsname
\let\ifluaotfloadloaded\endinput

\expandafter\ifx\csname ProvidesPackage\endcsname\relax
  \input luatextra.sty
\else
  \NeedsTeXFormat{LaTeX2e}
  \ProvidesPackage{luaotfload}%
    [2010/01/08 v1.07 ConTeXt font loading system]
  \RequirePackage{luatextra}
\fi

\expandafter\edef\csname otfl@AtEnd\endcsname{%
  \catcode64 \the\catcode64\relax
}

\catcode64 11

%    \end{macrocode}
%
%    The attributes are allocated here. The |otfl@| prefix is added to prevent
%    name collision.
%
%    \begin{macrocode}

\newluatexattribute\otfl@state
\newluatexattribute\otfl@markbase
\newluatexattribute\otfl@markdone
\newluatexattribute\otfl@markmark
\newluatexattribute\otfl@cursbase
\newluatexattribute\otfl@curscurs
\newluatexattribute\otfl@cursdone
\newluatexattribute\otfl@kernpair
\newluatexattribute\otfl@color

%    \end{macrocode}
%
%    Two small macros to register or unregister the callbacks. Without the
%    callbacks this package is totally turned off.
%
%    \begin{macrocode}

\def\otfl@off{
\luadirect{luaotfload.unregister_callbacks()}
}

\def\otfl@on{
\luadirect{luaotfload.register_callbacks()}
}

%    \end{macrocode}
%
%    We load the |lua| file, and we turn the package on.
%
%    \begin{macrocode}

\luatexUseModule{luaotfload}

\otfl@on

\otfl@AtEnd
%    \end{macrocode}
% \iffalse
%</package>
% \fi
% \Finale
\endinput
